\documentclass[UTF8]{article}
\usepackage[letterpaper,top=2cm,bottom=2cm,left=2cm,right=2cm,marginparwidth=1.75cm]{geometry}
\usepackage{amssymb}
\usepackage{amsmath}
\usepackage{zhnumber}
\usepackage{ctex}
\title{2022~2023线性代数I (H) 期末答案 }
\author{张晋恺}
\date{2024年1月18日}
\begin{document}
\maketitle
    
 一、
     $\begin{bmatrix}  1&k&1&1\\1&-1&1&1\\k&1&2&1  \end{bmatrix}\rightarrow$ 
     $\begin{bmatrix}  1&k&1&1\\0&k+1&0&0\\0&0&2-k&1-k  \end{bmatrix}$
     当 $k=2$ 时无解。
     当$k+1=0$时,即
     $k=-1$
     时为
     $\begin{Vmatrix}  1&-1&1&1\\0&0&3&2  \end{Vmatrix}$
     令$x_1=t$有通解${(0 \ -\frac{1}{3} \ \frac{2}{3})}^T+t{(1 \ 1 \ 0)}^T$
     当$k\neq 2$且$k\neq -1$时有特解$(\frac{1}{2-k} \ 0  \ \frac{1-k}{2-k})$
\par
二、
     $f(x_1,x_2,x_3)={(x_1+x_2)}^2-{(x_2-x_3)}^2+0{x_3}^2$
     令$\begin{bmatrix}u\\v\\w\end{bmatrix}= 
     \begin{bmatrix}1&1&0\\0&1&-1\\0&0&1\end{bmatrix}\begin{bmatrix}x_1\\x_2\\x_3\end{bmatrix}$
     \par
     故$\begin{bmatrix}x_1\\x_2\\x_3\end{bmatrix}= {\begin{bmatrix}1&1&0\\0&1&-1\\0&0&1\end{bmatrix}}^{-1}\begin{bmatrix}u\\v\\w\end{bmatrix}
     ={\begin{bmatrix}1&-1&-1\\0&1&1\\0&0&1\end{bmatrix}}\begin{bmatrix}u\\v\\w\end{bmatrix}$
     \par
     正负惯性指数均为1
\par
三、
     $A{A}^*={\left\lvert A\right\rvert}E$,故$A=(A^*)^{-1}*\left\lvert A\right\rvert $
     $|A^*|=|A|^2=16,|A|=±4$,
     $(A^*)^{-1}=\begin{bmatrix} \frac{1}{2}&\frac{1}{4}&\frac{1}{4} \\ \frac{1}{4}&\frac{1}{2}&\frac{1}{4}\\\frac{1}{4}&\frac{1}{4}&\frac{1}{2}  \end{bmatrix}$
\par
     故$A=\pm\begin{bmatrix}2&1&1\\1&2&1\\1&1&2 \end{bmatrix}$
\par
     四、$LALU$例$4.3$
\par
五、
     容易知道A的特征值为1,2,-1.

     $|A+3E| = \prod \lambda_i' = \prod (\lambda_{i}+3) = 4 \times 5 \times 2 = 40$
     \par
\par
六、
\par
   (1)$r(A)=r$,通过可逆矩阵$P$进行列变换作用$A$可以把其后$n-r$列变为0,再通过$P^{-1}$进行行变换即可 
\par 
    (2)相抵分解
   $A=P \left(\begin{matrix}1&0\\0&0\end{matrix}\right) Q=\alpha\beta^T,tr(A)=1$
   故$A^2=\alpha\beta^T\alpha\beta^T=tr (A)\alpha\beta^T=A$

\par
七、
\par
   (1)验证$\sigma (\lambda p_1(x)+\mu p_2(x))=\lambda p_1(x)+\mu p_2(x)$即可
\par
   (2) 取$R_3[x]$的基$B_1=\{1 \ x \  x^2\}=\{e_1 \ e_2\ e_3 \}$
   \par
   取$R^{2\times2}$
   的基为$B_2=\{\left(\begin{matrix}1&0\\0&0\end{matrix}\right) \ \left(\begin{matrix}0&1\\0&0\end{matrix}\right) \ \left(\begin{matrix}0&0\\1&0\end{matrix}\right) \ 
   \left(\begin{matrix}0&0\\0&1\end{matrix}\right) \}=\{\epsilon_1 \ \epsilon_2 \ \epsilon_3 \ \epsilon_4 \}$
   \par
   $\sigma(e_1 \ e_2\ e_3 )=(\epsilon_1 \ \epsilon_2 \ \epsilon_3 \ \epsilon_4 )\left(\begin{matrix}0&-1&-3\\0&0&0\\0&0&0\\1&0&0\end{matrix}\right)$
   \par
   (3)$Im\sigma=span\{\epsilon_3 \ \epsilon_4\}$
      设$p(x)=ax^2+bx+c\in \ker\sigma\subseteq R_3[x]$
      则有$\begin{cases}(a+b+c)-(4a+2b+c)=0\\c=0\end{cases}$
      \par
      解得$\ker\sigma=span\{-3x^2+x\}$
    \par
    (4)给出维数相同的子空间即可,例如$span\{1 \ x\}\cong Im\sigma,span\{\epsilon_1\}\cong \ker\sigma$
\par
八、
\par
   (1)错,令所有的$\alpha$为零向量,$\beta$非零即可
   \par
   (2)错,平面上三线共点的两两直线均可张成平面,但是任意两条直线显然线性无关
   \par
   (3)错,反例:$\begin{bmatrix}1&0\\0&0\end{bmatrix}$
   \par
   (4)显然是正确的

\end{document}

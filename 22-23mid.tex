\documentclass[UTF8]{article}
\usepackage[letterpaper,top=2cm,bottom=2cm,left=2cm,right=2cm,marginparwidth=1.75cm]{geometry}
\usepackage{amssymb}
\usepackage{amsmath}
\usepackage{mismath}
\usepackage{zhnumber}
\usepackage{ctex}
\title{2022--2023谈之奕线性代数I (H) 期中答案 }
\author{张晋恺}
\date{2024年1月25日}
\begin{document}
\maketitle
\leftline{一、}
初等行变换可化为$ \left(\begin{matrix}  2&-4&5&3&1\\0&0&-7&-5&1\\0&0&0&0&\lambda-3 \end{matrix}\right)$
故当$\lambda=3$时有解,解得其一般解为\\
$$\left(\begin{matrix} \dfrac{4}{5}\\0\\0\\-\dfrac{1}{5}\end{matrix}\right)+k_2\left(\begin{matrix} 2\\1\\0\\0\end{matrix}\right)
+k_3\left(\begin{matrix} -\dfrac{2}{5}\\0\\1\\-\dfrac{7}{5}\end{matrix}\right)$$
\par
\leftline{二、}
取$\alpha_2,\alpha_3,\alpha_4,\alpha_5$即可,按列排成矩阵化成简化阶梯矩阵说明或者利用行列式不为0.\par
\leftline{三、}
(1)$x_1=x_2,\Rightarrow \ker\sigma=\spa\{ ( 1,1 )\}$,令$\tau=I-\sigma$,则$\tau ( x_1,x_2 )= (x_2,3x_2-2x_1 )$
取自然基可得
$$\im \sigma= \spa \{(0,-2),(1,3)\}$$.\par
(2)该结论不一定成立,证明如下:
$$
\ker\tau\subseteq im(I-\tau)\Rightarrow\forall\alpha\in \ker\sigma,\exists\beta,s.t,(I-\tau)(\beta)=\alpha,\\
\implies I(\beta)-\tau(\beta)=\alpha,\tau(\beta)-\tau^2(\beta)=\tau(\alpha)=0,thus,\tau(\beta)=\tau^2(\beta)
$$
故只要证$(I-\tau)(\tau(\gamma))=0\implies\tau(\gamma)=\tau^2(\gamma)$但是,$\gamma$要求是任意的,而$\beta$只是特定的,并不能保证覆盖整个V,事实上第一问的$\tau$就是一个反例.
\par
\leftline{四、}
(1)证明略,题目应改为次数小于3
设$$f(x)=ax^2+bx+c,f(1)=0\Rightarrow a+b+c=0$$解得$$(a,b,c)=k_1(1,0,-1)+k_2(0,1,-1)$$\\
故$$W=\spa \{x^2-1,x-1\},\dim W=2$$.\par
(2)
证明略,$\ker T$即为第一问的W,$\im T$=$\spa\{1\}$.
\par
(3) 二维线性空间任意三个向量线性相关,$f,g,h \in W$ 故线性相关.
\par
\leftline{五、}
(1)错,$\alpha=0,\beta \neq 0$为反例.\par
(2)错,对于复数域不构成线性空间,因为实数与复数的数乘不封闭于实数.\par
(3)错,三元向量,$(0,0,1)+(1,0,0)$不封闭于$\mathbf{R^3_1}$.\par
(4)错,反例与第一问一致.\par
(5)错,不一定是满射.\par
(6)对,将$W$的基扩充为$V$的基,构造线性映射将W的基映射为0,扩充的基做恒等变换即可得到要求的线性映射,\\
即$\sigma\{\alpha_i,\beta_j\}=\{0,\beta_j\}$,对于$\gamma=\Sigma \lambda_i\alpha_i+\Sigma\mu_j\beta_j\in\ker\sigma$易得到$\mu_j=0$,故$\ker\sigma=W$.


\end{document}